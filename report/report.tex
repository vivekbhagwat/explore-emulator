\documentclass[11pt]{article}
\setlength{\oddsidemargin}{0pt}
\setlength{\textwidth}{470pt}
% \usepackage{amssymb}
\usepackage{amsfonts}
% \usepackage{amsmath}
\usepackage{latexsym}
% \usepackage{epsfig}
\usepackage{listings}

\title{Group-based venue recommendations based on foursquare checkins}

\author{Vivek Bhagwat\\
{\tt vsb2110@columbia.edu}
}

\begin{document}

\maketitle

\section{Abstract}
When in a new area with a group of friends, it can be difficult to figure out what restaurant, bar, or caf\'e would be the best to go to ---
and which place you would like the most. Foursquare has a rich set of data about venues and data about where individuals like to go. 
Using this individual's history and information about the venue, a simple recommendation engine can be built. While foursquare 
has built a recommendation engine called ``explore,'' this system has no ability to analyze data specific to the group. However,
these data are not necessarily hard to collect --- it simply requires all users involved supplying access to their checkin history.
Once these data are acquired, similar techniques can be used to training on a group's data as an individual's.

\section{Data}
Foursquare has an easy-to-use API which allows access to what venues are nearby, checkin history, and information about those venues. 
The data used for this recommendation engine included users' checkin history, information about the category or categories of the
venues, how popular each venue is, and which venues are nearby.\\
\\
However, there exist certain key limitations to this recommendation system. Due to privacy concerns, foursquare does not provide access
to the checkin history of friends. Without access to friends' checkin history, the amount of data
on which a system can be trained is limited. This is a key advantage of foursquare's explore function --- which does have access to such
data. Given access to this data, the recommendation engine could easily train on significantly more data, because the system would 
essentially train on the extended circles of the group. Despite this, however, the group training does prove to take into account enough 
data, provided all participants have checked in to many different types of venues --- so as to be able to provide data on any given
query that is searched.\\
\\
Another limitation with the data is that foursquare data almost exclusively provides one category per venue. This severely limits the more
granular information that exists about venues – such as what type of bar it might be, whether a restaurant is a casual one or not, and other
pertinent information such as price range. Information like this might be scattered in tips for a venue, but the entity extraction involved 
would render the system needlessly complex for very simple and very fast recommendations.\\
\\
Another limitation with the data is that evaluating metrics on the accuracy of the recomendations is not as simple as splitting the training
data into a train/test split. This is because the amount an individual ``likes'' a venue is not available through data foursquare provides,
except again through analysis of tips that user has left, which are often unreliable sources of information. As a result, it is difficult to
evaluate whether a recommendation is a ``good'' recommendation or not. The metric of choice therefore becomes subjective, and based on 
human prediction.

\section{System}
The foursquare venue data provides information about the location of the venue, the category or categories it falls under, and statistics
about the number of checkins. For the limited purposes of this project, the OAuth token necessary to obtain data for each user was 
obtained manually for each of the four users who gave permission to use their checkin data anonymously. Were this project to become 
more of a product, this process would have to be automated (which is indeed possible, but unnecessary for these purposes).\\ 
\\
To train the system, the checkin history for each user is first downloaded. A feature vector and weights are then created, with a simple
count of each category's $\tt{shortName}$ function defining the weights.



\end{document}

